\documentclass[11pt, a4paper]{article}

% --- OPTIMIZATION SETTINGS ---
\usepackage{mathptmx} % Lightweight font (Times)
\usepackage[T1]{fontenc}
\usepackage[varqu,varl]{zi4} % Consolas-like monospaced font

% "final" forces images to show. Use "draft" if you need to debug text only.
\usepackage[final]{graphicx} 

% --- STANDARD PACKAGES ---
\usepackage{amsmath}
\usepackage{geometry}
\usepackage{booktabs}
\usepackage{fancyhdr}
\usepackage{listings}
\usepackage{xcolor}
\usepackage{multirow}
\usepackage{adjustbox}
\usepackage{float}
\usepackage{siunitx}
\usepackage{hyperref}

% --- CONFIGURATION ---
\setlength{\emergencystretch}{2em}
\geometry{left=25mm, right=25mm, top=25mm, bottom=25mm}

\sisetup{
  parse-numbers=false,
  round-mode=figures,
  round-precision=4
}

\lstset{
  basicstyle=\ttfamily\small,
  breaklines=true,
  frame=single,
  columns=fullflexible,
  captionpos=b
}

\title{Problem 1: Cache Hierarchy Optimization using gem5}
\author{Rahate Tanishka Shivendra (22CS30044), Shivam Choudhury (22CS10072)}
\date{}

\begin{document}

\maketitle

\section{Introduction}
This report presents a comprehensive investigation into cache hierarchy performance using the gem5 simulator. We analyzed a memory-intensive matrix multiplication benchmark across a wide range of cache configurations to understand the impact of L1/L2 cache sizes and associativities on execution time, cache hit rates, and overall system performance. 

To provide a thorough analysis, we extended the assignment by testing three different matrix sizes (64$\times$64, 128$\times$128, and 256$\times$256), enabling us to observe how cache performance scales with working set size. This multi-scale approach reveals critical insights into cache design trade-offs for different application characteristics.

\section{Part 1: Environment Setup}

\subsection{gem5 Configuration}
The gem5 environment was configured with a RISC-V build using the TimingSimpleCPU model. The core of our simulation infrastructure is the \texttt{cache\_config.py} script, which defines a three-level cache hierarchy (L1 instruction, L1 data, and unified L2) with parameterized sizes and associativities.

\subsection{Cache Configuration Script}
The \texttt{cache\_config.py} script provides command-line arguments for flexible cache configuration:

\begin{lstlisting}[language=Python, caption=Key sections of cache\_config.py]
import m5
from m5.objects import *
import argparse

parser = argparse.ArgumentParser()
parser.add_argument("--l1d_size", type=str, default="64kB")
parser.add_argument("--binary", type=str, required=True)
parser.add_argument("--l2_size", type=str, default="256kB")
parser.add_argument("--l1_assoc", type=int, default=2)
parser.add_argument("--l2_assoc", type=int, default=8)

# Cache hierarchy with L1I, L1D, and unified L2
system.cpu.icache = L1_ICache(args)
system.cpu.dcache = L1_DCache(args)
system.l2cache = L2Cache(args)
\end{lstlisting}

\subsection{Matrix Multiplication Benchmark}
The benchmark uses a standard triple-nested loop matrix multiplication with configurable matrix size:

\begin{lstlisting}[language=C, caption=Matrix size configuration in matrix\_multiply.c]
#ifndef MATRIX_SIZE
#define MATRIX_SIZE 128  /* Default: 128. Override with -DMATRIX_SIZE=64 */
#endif

volatile int A[MATRIX_SIZE][MATRIX_SIZE];
volatile int B[MATRIX_SIZE][MATRIX_SIZE];
volatile int C[MATRIX_SIZE][MATRIX_SIZE];
\end{lstlisting}

\subsection{Successful Test Run}
A test run was performed with the default cache configuration to verify the setup:

\begin{lstlisting}[language=bash]
./build/RISCV/gem5.opt configs/cache_config.py \
    --l1d_size=16kB --l2_size=256kB \
    --l1_assoc=4 --l2_assoc=8 \
    --binary=benchmarks/matrix_multiply
\end{lstlisting}

\begin{figure}[H]
    \centering
    \includegraphics[width=0.9\textwidth]{gem5_test_run.png}
    \caption{Screenshot of successful gem5 test run showing simulation initialization and completion.}
    \label{fig:test_run}
\end{figure}

\subsection{Cache Statistics Output}
The gem5 simulator generates detailed statistics in \texttt{stats.txt}. Key metrics from a sample run confirm that the cache hierarchy is functioning correctly, with the L1 data cache achieving a 99.1\% hit rate.

\section{Part 2: Single Parameter Sweep}
\begin{table}[htbp]
\centering
\caption{Summary Statistics Across Matrix Sizes}
\label{tab:summary_stats}

\scriptsize
\setlength{\tabcolsep}{3pt}
\renewcommand{\arraystretch}{1.1}

\adjustbox{max width=\textwidth}{
\begin{tabular}{
  ll
  S[table-format=1.2e-2]
  S[table-format=1.2e-2]
  S[table-format=1.2e-2]
  S[table-format=1.2e-2]
  S[table-format=3.0]
}

\toprule
\textbf{Matrix} & \textbf{Metric} &
{Mean} & {Median} & {Std} & {Min} & {Max} \\
\midrule

\multirow{5}{*}{64$\times$64}
& simSec   & 1.10e-02 & 1.06e-02 & 4.22e-04 & 1.06e-02 & 1.16e-02 \\
& hostSec  & 6.08     & 6.82     & 1.35     & 3.56     & 8.92     \\
& L1 miss  & 2.66e-02 & 5.13e-03 & 2.61e-02 & 3.81e-03 & 7.24e-02 \\
& L2 miss  & 4.14e-01 & 5.89e-01 & 3.21e-01 & 4.63e-02 & 7.85e-01 \\
& Cache(kB)& 336      & 288      & 162      & 144      & 576      \\
\midrule

\multirow{5}{*}{128$\times$128}
& simSec   & 1.14e-01 & 1.29e-01 & 2.30e-02 & 8.03e-02 & 1.31e-01 \\
& hostSec  & 53.9     & 64.2     & 15.1     & 27.6     & 76.3     \\
& L1 miss  & 3.44e-01 & 4.96e-01 & 2.22e-01 & 2.52e-02 & 5.07e-01 \\
& L2 miss  & 1.56e-02 & 3.45e-03 & 2.05e-02 & 1.96e-03 & 6.94e-02 \\
& Cache(kB)& 336      & 288      & 162      & 144      & 576      \\
\midrule

\multirow{5}{*}{256$\times$256}
& simSec   & 1.46     & 1.07     & 0.58     & 1.02     & 2.28     \\
& hostSec  & 521      & 523      & 109      & 294      & 734      \\
& L1 miss  & 5.02e-01 & 5.00e-01 & 2.59e-03 & 5.00e-01 & 5.05e-01 \\
& L2 miss  & 3.48e-01 & 4.18e-02 & 4.62e-01 & 1.43e-03 & 1.00      \\
& Cache(kB)& 336      & 288      & 162      & 144      & 576      \\
\bottomrule
\end{tabular}
}

\end{table}

Table~\ref{tab:summary_stats} summarizes the statistical distribution of execution time, cache miss rates, and total cache capacity across all tested matrix sizes.

\subsection{Methodology}
For the single parameter sweep, we investigated the impact of \textbf{L2 cache size} while keeping other parameters constant. We chose L2 cache size because it represents a critical trade-off point in cache hierarchy design.

\textbf{Sweep Configuration:}
\begin{itemize}
    \item \textbf{Variable Parameter:} L2 cache size (128kB, 256kB, 512kB)
    \item \textbf{Fixed Parameters:} L1D = 16kB (assoc=4), L1I = 16kB (assoc=4), L2 assoc = 8
    \item \textbf{Matrix Sizes Tested:} 64$\times$64, 128$\times$128, 256$\times$256
\end{itemize}

\subsection{Results and Analysis}

\begin{figure}[H]
    \centering
    \includegraphics[width=0.8\textwidth]{l2_size_vs_time.png}
    \caption{Impact of L2 Cache Size on Execution Time (log scale) with L1 cache fixed at 16kB.}
    \label{fig:l2_sweep}
\end{figure}

\paragraph{Observed Trends:}
As shown in Figure \ref{fig:l2_sweep}, the relationship between L2 cache size and execution time exhibits distinct patterns across different matrix sizes:

\begin{itemize}
    \item \textbf{64$\times$64 Matrix:} Performance is relatively insensitive to L2 size. The working set fits comfortably within even the smallest L2 cache tested.
    \item \textbf{128$\times$128 Matrix:} A moderate performance improvement is observed when increasing L2 from 128kB to 256kB.
    \item \textbf{256$\times$256 Matrix:} The most dramatic performance cliff occurs between 128kB and 256kB L2 cache. Execution time drops by approximately 40\% when doubling the L2.
\end{itemize}

\paragraph{Why This Effect Occurs:}
When the L2 cache is too small to hold the working set, \textbf{capacity misses} dominate, forcing frequent main memory accesses. Each main memory access incurs a penalty of 100+ cycles, compared to 10-20 cycles for L2 hits.

\section{Part 3: Multi-Parameter Analysis}

\subsection{Full Parameter Sweep Methodology}
We conducted a comprehensive multi-parameter sweep across the following dimensions:
\begin{itemize}
    \item \textbf{L1D Cache Size:} 16kB, 32kB, 64kB
    \item \textbf{L2 Cache Size:} 128kB, 256kB, 512kB
    \item \textbf{L1 Associativity:} 2, 4, 8
    \item \textbf{L2 Associativity:} 4, 8, 16
    \item \textbf{Matrix Sizes:} 64$\times$64, 128$\times$128, 256$\times$256
\end{itemize}

This resulted in 108 unique cache configurations per matrix size (324 total simulations).

\subsection{Impact of L1 Cache Size}

\begin{figure}[H]
    \centering
    \includegraphics[width=0.9\textwidth]{l1_size_vs_time.png}
    \caption{Impact of L1 Cache Size on Execution Time (log scale) with L2 fixed at 256kB.}
    \label{fig:l1_time}
\end{figure}

The L1 cache serves as the primary mechanism for reducing effective memory access time (EMAT). As observed in Figure \ref{fig:l1_time}, the performance scaling is non-linear across matrix sizes. For the 128$\times$128 matrix, a significant performance improvement occurs between 32kB and 64kB, suggesting that 64kB is the critical threshold.

\subsection{Impact of Associativity}

\begin{figure}[H]
    \centering
    \includegraphics[width=\textwidth]{l1_miss_rate_by_assoc.png}
    \caption{L1 Miss Rate by Associativity, faceted by Matrix Size.}
    \label{fig:l1_miss_rate}
\end{figure}

Associativity is designed to mitigate conflict misses. Figure \ref{fig:l1_miss_rate} reveals that for 128$\times$128 and 256$\times$256 matrices, miss rates remain high across all associativity levels for smaller cache sizes. This is a hallmark of \textbf{capacity misses}—when data simply doesn't fit, increasing associativity offers no relief.

\subsection{Cache Efficiency Heatmap}

\begin{figure}[H]
    \centering
    \includegraphics[width=0.7\textwidth]{heatmap_time_128x128.png}
    \caption{Execution Time Heatmap for 128$\times$128 matrix showing the relationship between L1 and L2 cache sizes.}
    \label{fig:heatmaps}
\end{figure}

The heatmap in Figure \ref{fig:heatmaps} reveals a critical insight: execution time is almost entirely \textbf{vertically stratified}. Changes in L1 size (Y-axis) result in nearly identical colors within each column, whereas changes in L2 size (X-axis) result in massive color shifts.

\subsection{Comprehensive Multi-Parameter Analysis}

\subsubsection{Simulation Ticks vs Cache Parameters}

\begin{figure}[H]
    \centering
    \includegraphics[width=0.7\textwidth]{simticks_vs_l2_size.png}
    \caption{Simulation ticks vs L2 cache size across all three matrix sizes.}
    \label{fig:simticks_analysis}
\end{figure}

Figure \ref{fig:simticks_analysis} shows how simulation ticks vary with L2 cache size. The 256$\times$256 matrix exhibits a dramatic performance cliff when L2 cache drops below 256kB, with simulation ticks nearly doubling.

\subsubsection{L2 Cache Hit Rate Analysis}

\begin{figure}[H]
    \centering
    \includegraphics[width=0.7\textwidth]{l2_hitrate_vs_l2_size.png}
    \caption{L2 hit rate vs L2 cache size across all matrix sizes.}
    \label{fig:l2_hitrate}
\end{figure}

The L2 cache hit rate is a critical metric for understanding memory hierarchy efficiency. Figure \ref{fig:l2_hitrate} reveals that the 128$\times$128 matrix achieves near-perfect L2 hit rates ($>$95\%) with 256kB or larger L2 caches.

\subsubsection{Multi-Parameter Interaction Grids}

\begin{figure}[H]
    \centering
    \includegraphics[width=\textwidth]{multi_param_grid_128x128.png} % Ensure extension matches your file
    \caption{Multi-parameter analysis grid for 128$\times$128 matrix.}
    \label{fig:multiparam}
\end{figure}

To visualize the complex interactions between all cache parameters and performance metrics, we generated 4$\times$4 analysis grids (Figure \ref{fig:multiparam}). These grids reveal that:
\begin{itemize}
    \item \textbf{L2 cache size} has the strongest correlation with all performance metrics.
    \item \textbf{L1 associativity} shows diminishing returns beyond 4-way for most workloads.
    \item \textbf{L2 associativity} has minimal impact when cache capacity is sufficient.
\end{itemize}

\subsection{Top Configurations}

Based on our comprehensive sweep, the top 3 configurations ranked by execution time for each matrix size are:

\begin{table}[htbp]
\centering
\caption{Top 3 Fastest Configurations per Matrix Size}
\begin{tabular}{@{}lccccr@{}}
\toprule
\textbf{Matrix} & \textbf{L1D Size} & \textbf{L2 Size} & \textbf{L1 Assoc} & \textbf{L2 Assoc} & \textbf{Exec Time (s)} \\
\midrule
64$\times$64 & 64kB & 256kB & 2 & 8 & 0.010588 \\
64$\times$64 & 64kB & 256kB & 2 & 16 & 0.010588 \\
64$\times$64 & 64kB & 256kB & 4 & 4 & 0.010588 \\
\midrule
128$\times$128 & 64kB & 512kB & 4 & 8 & 0.080250 \\
128$\times$128 & 64kB & 512kB & 4 & 16 & 0.080250 \\
128$\times$128 & 64kB & 512kB & 4 & 4 & 0.080250 \\
\midrule
256$\times$256 & 32kB & 512kB & 8 & 16 & 1.017607 \\
256$\times$256 & 32kB & 512kB & 8 & 8 & 1.017607 \\
256$\times$256 & 32kB & 512kB & 4 & 16 & 1.017607 \\
\bottomrule
\end{tabular}
\end{table}

\section{Part 4: Design Analysis \& Recommendations}

\subsection{(a) Performance Bottlenecks}

\paragraph{Question:} Is execution time dominated by L1D misses, L2 misses, or memory stalls?

\paragraph{Answer:}
The performance bottleneck varies significantly with matrix size:
\begin{itemize}
    \item \textbf{64$\times$64 Matrix:} Not bottlenecked by cache. Less than 0.1\% of memory requests reach main memory.
    \item \textbf{128$\times$128 Matrix:} Moderately affected by L1D misses, but L2 effectively filters them.
    \item \textbf{256$\times$256 Matrix:} Heavily dominated by L2 misses. When L2 cache is insufficient ($<$256kB), the L2 miss rate can reach 100\%.
\end{itemize}

\textbf{Key Insight:} For large working sets, L2 capacity is the critical bottleneck.

\subsection{(b) Cache Efficiency}

\paragraph{Question:} Which cache level has the best hit rate? Why? Is L2 size or associativity more important?

\paragraph{Answer:}
\textbf{L2 Size vs. Associativity:} Our data conclusively shows that \textbf{L2 size is far more important than associativity}. Doubling L2 size from 256kB to 512kB improves performance by 20-40\% for large matrices, while doubling associativity improves it by $<$2\%.

\subsection{(c) Cost-Benefit Trade-off}

\paragraph{Question:} What's the smallest L1D+L2 configuration that achieves 90\% of peak performance?

\paragraph{Answer:}
\begin{table}[htbp]
\centering
\caption{Minimal Configurations for 90\% Peak Performance}
\begin{tabular}{@{}lcccc@{}}
\toprule
\textbf{Matrix} & \textbf{L1D} & \textbf{L2} & \textbf{Total} & \textbf{Performance vs. Peak} \\
\midrule
64$\times$64 & 16kB & 128kB & 144kB & 99.2\% \\
128$\times$128 & 32kB & 256kB & 288kB & 96.8\% \\
256$\times$256 & 16kB & 512kB & 528kB & 99.5\% \\
\bottomrule
\end{tabular}
\end{table}

\subsection{(d) Design Recommendations}

\begin{table}[htbp]
\centering
\caption{Recommended Cache Configurations by Design Goal}
\begin{tabular}{@{}lccccl@{}}
\toprule
\textbf{System Type} & \textbf{L1D} & \textbf{L2} & \textbf{L1 Assoc} & \textbf{L2 Assoc} & \textbf{Justification} \\
\midrule
Power-Constrained & 16kB & 128kB & 2 & 4 & Minimal area/power \\
High-Performance & 64kB & 512kB & 8 & 16 & Maximum performance \\
Balanced & 32kB & 256kB & 4 & 8 & Best perf/cost ratio \\
\bottomrule
\end{tabular}
\end{table}

\section{Part 5: Pareto Optimality Analysis}

In this section, we identify the \textbf{Pareto-optimal} configurations for our cache hierarchy. A configuration is Pareto-optimal if no other configuration can improve one metric (e.g., execution time) without degrading another (e.g., cache size/cost).

\begin{figure}[H]
    \centering
    \includegraphics[width=0.95\textwidth]{pareto_frontier.png}
    \caption{Design Space Exploration showing all 108 configurations per matrix size (shadowed points) and the resulting Pareto Front (solid lines).}
    \label{fig:pareto}
\end{figure}

\subsection{Pareto Front Characteristics}
As shown in Figure \ref{fig:pareto}, the Pareto front shifts upward and to the right as matrix size increases.
\begin{itemize}
    \item \textbf{High-Efficiency Zone:} There is a distinct "knee" where doubling cache size yields massive performance gains.
    \item \textbf{Diminishing Returns Zone:} Beyond 256kB for small matrices and 512kB for large ones, increasing cache size (cost) yields negligible performance improvements.
\end{itemize}

\section{Conclusion}

This comprehensive cache hierarchy study reveals several critical insights for processor design:

\begin{enumerate}
    \item \textbf{L2 capacity dominates performance} for memory-intensive workloads with large working sets.
    \item \textbf{Working set size is the primary determinant} of optimal cache configuration.
    \item \textbf{Associativity shows diminishing returns} beyond 4-way for L1 and 8-way for L2.
    \item \textbf{The "balanced" configuration} (32kB L1D, 256kB L2, moderate associativity) represents the sweet spot for general-purpose computing.
\end{enumerate}

\end{document}