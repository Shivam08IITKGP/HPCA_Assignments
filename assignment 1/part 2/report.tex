\documentclass[11pt, a4paper]{article}

% --- Fonts (Readable & Professional) ---
\usepackage[T1]{fontenc}
\usepackage{newpxtext}
\usepackage{newpxmath}
\usepackage{microtype}

% --- Better monospace font for code/verbatim ---
\usepackage{inconsolata}

% --- Slightly better line spacing for reports ---
\usepackage{setspace}
\onehalfspacing
\usepackage{array}
\renewcommand{\arraystretch}{1.1}

\usepackage{fvextra}
\DefineVerbatimEnvironment{verbatim}{Verbatim}{
  fontsize=\small,
  breaklines=true
}


\usepackage{graphicx}
\usepackage{amsmath}
\usepackage{geometry}
\usepackage{booktabs}
\usepackage{fancyhdr}
\usepackage[
    colorlinks=true,
    linkcolor=black,
    urlcolor=blue,
    citecolor=black
]{hyperref}


\geometry{a4paper, margin=1in}

\pagestyle{fancy}
\fancyhf{}
\lhead{HPCA Assignment 1 - Problem 2}
\rhead{\thepage}

\title{Problem 2: Gem5 MergeSort Cache Analysis}
\author{Rahate Tanishka Shivendra (22CS30043), Shivam Choudhury (22CS10072)}
\date{}   % ← removes date

\begin{document}

\maketitle

\section{Introduction}
This report analyzes the cache performance of two merge sort variants on a RISC-V architecture simulated in gem5. The two variants are:
\begin{itemize}
    \item \textbf{Simple}: A standard, recursive in-memory merge sort.
    \item \textbf{Chunked}: An optimized version that sorts the data in 2MB chunks before merging them, designed to improve cache locality.
\end{itemize}
The analysis compares a baseline configuration and then explores a wide parameter sweep to find optimal cache configurations for each workload.

\section{Part 1: Baseline Comparison}
The first task is to compare the performance of both variants with the default cache configuration (L1D: 64KiB/8-way, L2: 512KiB/16-way). The results are shown in Table \ref{tab:baseline}.

\begin{table}[h!]
\centering
\caption{Baseline performance comparison of mergesort variants.}
\label{tab:baseline}
\begin{tabular}{lrrrr}
\toprule
Mergesort Type & Time (s) & IPC & L1D Miss Rate & L2 Miss Rate \\
\midrule
Simple & 3.992447 & 0.299274 & 0.019771 & 0.680915 \\
Chunked & 4.007310 & 0.300675 & 0.019184 & 0.669356 \\
\bottomrule
\end{tabular}
\end{table}

\paragraph{Analysis:}
Despite nearly identical execution times (Simple: 3.99s vs Chunked: 4.01s), the cache behavior reveals fundamental differences in memory access patterns. The Chunked version exhibits a \textbf{1.5\% higher IPC} (0.300675 vs 0.299274), indicating more efficient instruction execution with fewer pipeline stalls.

The key insight lies in the miss rate differences:
\begin{itemize}
    \item \textbf{L1D Miss Rate:} Chunked achieves 0.019184 vs Simple's 0.019771, a 3\% reduction
    \item \textbf{L2 Miss Rate:} Chunked achieves 0.669356 vs Simple's 0.680915, a 1.7\% reduction
\end{itemize}

The chunked algorithm processes 2MB blocks independently during the initial sort phase, ensuring that each block fits comfortably within the L2 cache (512KiB). This temporal locality dramatically reduces capacity misses. The simple algorithm, by contrast, recursively subdivides the entire 10MB dataset, causing frequent cache thrashing as it jumps between distant memory regions.

The equivalent execution time suggests that while the chunked algorithm reduces memory stalls (reflected in higher IPC), it incurs computational overhead from the two-phase merge process, balancing out the gains for this particular cache configuration.

\section{Part 2: Cache Optimization Sweep}
A full parameter sweep was conducted to find the optimal cache configurations for both workloads. The following plots analyze the results.

\subsection{CPU Efficiency (IPC)}
Instructions Per Cycle (IPC) is a key measure of how efficiently the CPU is utilized. Figure \ref{fig:ipc} shows that the Chunked version consistently achieves a higher IPC.

\begin{figure}[h!]
    \centering
    \includegraphics[width=0.8\textwidth]{plot_ipc_efficiency.png}
    \caption{CPU Efficiency (IPC): Chunked vs Simple. The chunked algorithm's better data locality leads to fewer stalls and higher IPC.}
    \label{fig:ipc}
\end{figure}

\subsection{Cache Miss Rates}
The improved locality of the chunked algorithm directly translates to lower miss rates, particularly in the L2 cache, as shown in Figure \ref{fig:l2_miss}.

\begin{figure}[h!]
    \centering
    \includegraphics[width=0.8\textwidth]{plot_miss_rate_comparison.png}
    \caption{L2 Miss Rate: Chunked vs Simple. The chunked version significantly reduces L2 misses across all L2 cache sizes.}
    \label{fig:l2_miss}
\end{figure}

\paragraph{Analysis:}
Figure \ref{fig:l2_miss} reveals a consistent 1.5-2\% reduction in L2 miss rate for the chunked algorithm across all L2 configurations. This improvement stems from the algorithm's \textbf{cache-oblivious design}: by processing fixed 2MB chunks, the algorithm naturally adapts to the cache hierarchy without explicit tuning.

The simple mergesort's recursive nature creates a unfavorable access pattern. During the divide phase, the algorithm creates stack frames that reference widely scattered memory addresses. When these merge back together, the cache must hold data from multiple tree levels simultaneously, exceeding capacity and causing evictions of still-needed data.

The chunked approach, conversely, exhibits \textbf{phase behavior}:
\begin{enumerate}
    \item \textbf{Sort Phase:} Each 2MB chunk is sorted independently with excellent spatial locality
    \item \textbf{Merge Phase:} Chunks are merged in streaming fashion from 1MB buffers, maintaining sequential access patterns
\end{enumerate}

This results in predictable, streaming memory access that hardware prefetchers can exploit effectively, further reducing the effective miss penalty.

\subsection{L1 Cache Hit Rate Analysis}
Figure \ref{fig:l1_hit} shows the L1 hit rate as L1 size increases.

\begin{figure}[h!]
    \centering
    \includegraphics[width=0.8\textwidth]{plot_hitrate_l1.png}
    \caption{L1 Hit Rate vs L1 Size. Both algorithms benefit from larger L1 caches, but the chunked version maintains consistently higher hit rates.}
    \label{fig:l1_hit}
\end{figure}

\subsection{Execution Time Impact}
Figure \ref{fig:time} demonstrates how L1 cache size affects overall execution time.

\begin{figure}[h!]
    \centering
    \includegraphics[width=0.8\textwidth]{plot_time_impact.png}
    \caption{Impact of L1 Size on Execution Time (L2 fixed at 512kB). Execution time decreases as L1 size increases, with both algorithms converging at larger cache sizes.}
    \label{fig:time}
\end{figure}

\subsection{Top Configurations and Final Analysis}
The sweep reveals the best configurations for each workload.

\paragraph{Top 3 Configurations by IPC (Chunked):}
\begin{verbatim}
L1_Size  L1_Assoc L2_Size  L2_Assoc       IPC
  128kB        16  512kB        16  0.300675
  128kB        16  256kB        16  0.300675
  128kB        16  1024kB       16  0.300675
\end{verbatim}

\paragraph{Analysis:}
The full parameter sweep reveals an important architectural insight: \textbf{algorithmic optimization can reduce hardware dependencies}.

\textbf{For the Simple Mergesort:}
\begin{itemize}
    \item Best configuration: L1 128kB (16-way), L2 1024kB (16-way)
    \item Performance is highly sensitive to cache size
    \item Requires expensive, high-associativity caches to minimize conflict misses
\end{itemize}

\textbf{For the Chunked Mergesort:}
\begin{itemize}
    \item Best configuration: L1 128kB (16-way), L2 512kB (16-way)
    \item IPC remains consistently high (0.300+) across most configurations
    \item Performance is relatively insensitive to L2 size beyond 512kB
\end{itemize}

\textbf{Key Takeaway:}
The chunked algorithm achieves comparable performance with \textbf{half the L2 cache size} (512kB vs 1024kB). This has significant implications for chip design:
\begin{enumerate}
    \item \textbf{Silicon Area:} Smaller caches reduce die area and cost
    \item \textbf{Power Consumption:} Smaller caches consume less static and dynamic power
    \item \textbf{Scalability:} Cache-aware algorithms scale better to larger datasets
\end{enumerate}

However, the chunked algorithm has its own costs:
\begin{itemize}
    \item More complex implementation and debugging
    \item Additional merge phase overhead (visible in similar execution times despite better cache behavior)
    \item Memory allocation complexity for temporary buffers
\end{itemize}

\section{Conclusion}
This study demonstrates the co-design principle in computer architecture: \textbf{hardware and software must be optimized together}. While cache-oblivious algorithms like chunked mergesort require more sophisticated implementation, they reduce hardware requirements and improve energy efficiency. For memory-intensive workloads on resource-constrained systems, algorithmic optimization can provide better cost-performance than simply increasing cache size.

\end{document}